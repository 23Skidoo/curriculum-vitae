\documentclass[margin,line]{res}

\usepackage[utf8]{inputenc}
\usepackage[english]{babel}
\usepackage[bookmarksnumbered, unicode, pdftex]{hyperref}

\oddsidemargin -.5in
\evensidemargin -.5in
\textwidth=6.0in
\itemsep=0in
\parsep=0in

\renewcommand{\namefont}{\LARGE\bf}

\begin{document}

\name{Mikhail Glushenkov \vspace*{.1in}}

\begin{resume}
\section{\sc Contact Information}
\vspace{.05in}
\begin{tabular}{@{}p{2in}p{4in}}
Fysikgränd 3D/102&
+46738067025\\
90731 Umeå, Sweden&
\href{mailto:mikhail.glushenkov@gmail.com}
{\texttt{mikhail.glushenkov@gmail.com}}\\
\end{tabular}

\section{\sc Work Experience}

{\bf Haskell.org} \hfill {\it Summer 2013, Google Summer of Code Student}\\
(Work in progress.) I'm adding support for module-level parallel builds to
Cabal, a system for building and packaging Haskell libraries and programs.

% My code is integrated into the development version of Cabal.

{\bf Haskell.org} \hfill {\it Summer 2012, Google Summer of Code Student}\\
Added support for sandboxed builds and isolated environments to
Cabal. Sandboxing partially solves the important ``Cabal hell'' problem. Code
ships with Cabal since version 1.18.

{\bf Haskell.org} \hfill {\it Summer 2011, Google Summer of Code Student}\\
Parallelised the standard \texttt{cabal-install} tool, making it possible to
build multiple Cabal packages in parallel. Code ships with Cabal since version
1.16.

{\bf Codedgers Inc.} \hfill {\it December 2007-May 2011, Software Engineer}\\
Worked on Morpher (\url{http://morpher.com/}), an LLVM-based code
obfuscator. Wrote a generic compiler driver (\texttt{llvmc}). Implemented a
dynamic software watermarking tool suite.

% {\bf Goha.ru} \hfill {\it 2006-2007, Newswriter and Community Manager}\\
% Responsibilities included journalistic work and management of one of the biggest
% Russian MMORPG communities.

% {\bf Misc. freelance work} \hfill {\it 2005-2007}\\
% Included writing a GUI forum client in wxPython and some simple OpenGL apps in
% C++.

\section{\sc Education}
{\bf Umeå University}, Umeå, Sweden\\
MSc(Eng) in Computer Science, December 2014 (expected)

{\bf Technical University of Denmark}, Copenhagen, Denmark\\
Types at Work Summer School (\url{http://typesatwork.imm.dtu.dk/}), August 2009

\section{\sc Portfolio}

{\bf ghc-parmake}, 2011-now \hfill \url{https://github.com/23Skidoo/ghc-parmake}\\
A parallel driver for the GHC Haskell Compiler. It can build a Haskell
program in parallel using multiple cores (à la \texttt{make -j}). Used as a
drop-in replacement for \texttt{ghc --make}.

{\bf Haskell Platform}, 2009-now \hfill \url{http://hackage.haskell.org/platform/}\\
The Haskell Platform is a single, standard Haskell distribution for every
system, in the form of a blessed library and tool suite for Haskell. I maintain
the Windows installer.

% http://goo.gl/67nKm
{\bf llvmc}, 2007-2011 \hfill \url{http://llvm.org/releases/2.9/docs/CompilerDriver.html}\\
A generic compiler driver, designed to be customisable and extensible. Generates
a C++ driver program from a declarative description. Used to be a part of the
LLVM project.

% {\bf brabantio}, 2012 \hfill \url{https://github.com/23Skidoo/brabantio}\\
% An implementation of Othello (reversi) in Ocaml. Uses alpha-beta pruning for
% game tree search.

{\bf Open source}\\ I worked on patches for various open source projects
including Doxygen, LLVM and Cabal. I'm
\href{https://github.com/23Skidoo/}{\texttt{23Skidoo}} on GitHub.

{\bf Blog}, 2011-now \hfill \url{http://coldwa.st/e/blog}

\section{\sc Skills}

{\bf Programming Languages:} Proficient in Haskell, C++, C, Python. Familiar
with Ocaml, Java, Assembly (x86/MIPS/AVR), Matlab, SQL, Scheme, Lua, UNIX shell. \\
{\bf Tools:} Emacs, Git, SVN, MSVC, NSIS, etc.\\
{\bf Platforms:} UNIX-like (preferable), Win32.

% \section{\sc Certifications}

% \href{http://www.brainbench.com/transcript.jsp?pid=5078355}
% {Brainbench Certificate in C++}, 2008-12-01.

\section{\sc Misc}

Right to work in EU and EFTA countries. Willing to relocate.

\end{resume}
\end{document}
